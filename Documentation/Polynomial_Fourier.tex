\documentclass[11pt,english]{article}

% page layout
\usepackage{geometry}
\geometry{verbose,letterpaper,tmargin=0.75in,bmargin=0.75in,lmargin=0.75in,rmargin=0.75in}
% text spacing
\usepackage{setspace}
\onehalfspacing

% set font families
%\renewcommand\rmdefault{cmr}
%\renewcommand\sfdefault{cmss}
%\renewcommand\ttdefault{cmtt}

% math symbols
\usepackage{amssymb,amsmath,bm}

% vector
\newcommand{\vc}[1]{\ensuremath{\bm{#1}}}

\usepackage{hyperref}

\begin{document}

\title{Fourier transform notes}
\author{Michael P. Mendenhall (Caltech)}
\maketitle

\section{Conventions}

For a function $f(x)$, $x \in \left[ -\frac{1}{2}, \frac{1}{2} \right]$, we define the Fourier transform by:
\begin{equation}
	\tilde f(k) \equiv \int_{-1/2}^{1/2} f(x) e^{-2\pi i k x} dx
\end{equation}

\section{Fourier transforms of polynomials}

For (multi-variate) polynomials $P(x_1,\cdots,x_n) = \sum_{k_1,\cdots,k_n} c_{k_1,\cdots,k_n} \prod_i x_i^{k_i}$,
	Fourier transforms over rectangular domains $x_i \in [a_i,b_i]$ may be calculated analytically.
Transforms may be calculated integrating out one dimension $x_i$ at a time,
	so only the behavior of univariate polynomials need be considered.
Starting from a polynomial $P(x)$ within cell $x \in [a,b]$,
	we first apply the change-of-variable $\frac{x-a}{b-a} -\frac{1}{2} \rightarrow x$
	for convenience, so the domain of interest is $x \in \left[ -\frac{1}{2}, \frac{1}{2} \right]$.
Then, we define Fourier terms for wavenumber $k$ by
\begin{equation}
	\tilde P(k) \equiv \int_{-1/2}^{1/2} \left[ \sum_m c_m x^m \right] e^{-2\pi i k x} dx
\end{equation}
This integration may be carried out by reducing each term in the polynomial
	(starting from the highest $m > 0$) to next lower order using integration by parts:
\begin{equation}
	\int_{-1/2}^{1/2} x^m e^{-2 \pi i k x} dx
		= \frac{-1}{2\pi i k} \left[ \frac{1-(-1)^m}{2^m} (-1)^k -  m \int_{-1/2}^{1/2} x^{m-1} e^{-2\pi i k x} dx \right]
	\label{eq:ft_xn}
\end{equation}
noting that
\begin{equation}
	\frac{1-(-1)^m}{2^m} = 
	\begin{cases}
		2^{(1-m)},		& \textrm{$m$ odd}		\\
		0,				& \textrm{$m$ even}
	\end{cases}
\end{equation}
Once the progression reaches $m=0$, we use
\begin{equation}
	\int_{-1/2}^{1/2} e^{-2\pi i k x} dx = \frac{\sin \pi k}{\pi k} 
\end{equation}
which vanishes for integer $k \neq 0$ (note, the formulae above do not depend on $k$ being an integer, though this is the common case).
For the special case of $k=0$, one needs instead:
\begin{equation}
	\int_{-1/2}^{1/2} x^m dx =
	\begin{cases}
		\frac{2^{-m}}{m+1},		& \textrm{$m$ even}		\\
		0,						& \textrm{$m$ odd}
	\end{cases}
\end{equation}

\section{Mirrored cell}

For $T_2$ calculations, we need Fourier transforms for a
	``mirrored'' cell configuration.
Let $f(x)$, $x \in \left[ -\frac{1}{2}, \frac{1}{2} \right]$ be any function.
Mirroring instead around $x=\frac{1}{2}$, define the ``mirrored'' function
\begin{equation}
	f^m(x) \equiv
	\begin{cases}
		f(x),		& x \leq \frac{1}{2}		\\
		f(1-x)		& x > \frac{1}{2}
	\end{cases}
\end{equation}
We then wish to calculate the Fourier terms
\begin{equation}
	\tilde {f^m}(k) \equiv \int_{-1/2}^{3/2} f^m(x) e^{\pi i k x} dx
	= \int_{-1}^{1} f^m \left( x + \frac{1}{2} \right) e^{\pi i k x}e^{\frac{k}{2} \pi i} dx
\end{equation}
Since $f^m \left( x + \frac{1}{2} \right)$ is symmetric around $x=0$,
	only the symetric component of $e^{\pi i k x}$ will contribute to the integral.
Thus, we may symmetrize $e^{\pi i k x} \rightarrow \frac{1}{2}\left(e^{\pi i k x}+e^{-\pi i k x}\right)$
and integrate over the half-region where $f^m = f$:
\begin{equation}
	\tilde {f^m}(k) = e^{\frac{k}{2} \pi i} \int_{-1}^{0} f \left( x + \frac{1}{2} \right)
	\left(e^{\pi i k x}+e^{-\pi i k x}\right) dx
\end{equation}
now, shifting bounds with the substitution $u = x+\frac{1}{2}$:
\begin{equation}
\tilde {f^m}(k) = e^{\frac{k}{2} \pi i}
	\Bigg[
		e^{-\frac{k}{2} \pi i} \underbrace{\int_{-1/2}^{1/2} f(x) e^{\pi i k x} dx}_{\tilde f \left( -\frac{k}{2} \right)}
		+e^{\frac{k}{2} \pi i} \underbrace{\int_{-1/2}^{1/2} f(x) e^{-\pi i k x} dx}_{\tilde f \left( \frac{k}{2} \right)}
	\Bigg]
\end{equation}
producing a formula in terms of the Fourier components $\tilde{f}\left( \pm \frac{k}{2} \right)$,
	which are also used in the $T_2$ formula.



\section{Other notes}

\subsection{Simplification of $S_{B_x x}$}

since FT results for $\pm l$ are complex conjugates:
\begin{equation}
	i^{-k} = (-1)^k i^k
\end{equation}
\begin{align}
S_{B_x x}(\omega) & = \sum_{l_x=odd~-\infty }^{\infty } -\frac{(i)^{l_x}2L_x}{\pi ^{2}l_x^{2}}
	p\left(\frac{l_x \pi}{L_x}, \omega \right)
	\frac{1}{V} \int_{-\frac{L_x}{2}}^{\frac{L_x}{2}}\int_{-\frac{L_y}{2}}^{\frac{L_y}{2}} \int_{-\frac{L_z}{2}}^{\frac{L_z}{2}}
	B_x(\mathbf{x} )\mathrm{e}^{\frac{i\pi l_x x}{L_x}}d^3\mathbf{x} \\
		& = \sum_{l_x = 1,3,5,\cdots}^{\infty} \frac{(-1)^{(l_x-1)/2} 4 L_x}{\pi ^{2}l_x^{2}}
	p\left(\frac{l_x \pi}{L_x}, \omega \right)
	\Im \left[ \frac{1}{V} \iiint_V
	B_x(\mathbf{x} )\mathrm{e}^{\frac{i\pi l_x x}{L_x}}d^3\mathbf{x} \right]
\end{align}

\subsection{$^3$He $p(q,\omega)$}

To avoid computation overflow in large exponential terms in $p(q,\omega)$ for $^3$He,
	we may employ the ``Fadeeva function'' $w(z)$,
\begin{equation}
	w(z) \equiv e^{-z^2} \operatorname{erfc}(-iz); \ \ z = i \sqrt{\frac{m}{2kT}} \frac{\lambda+i\omega}{q}
\end{equation}
available as, e.g., \texttt{scipy.special.wofz(z)}.

\subsection{neutron $p(q,\omega)$}
$p(q,\omega)$ for UCN is written more simply by collecting terms in a dimensionless variable $x$:
\begin{equation}
	p(q,\omega) = \frac{3ix^2}{2\omega} \left[ \left(x - \frac{1}{x}\right)(2\operatorname{atanh}(x)-i\pi) - 2 \right],\ \
	x \equiv \frac{\omega}{\nu_\textrm{max}|q|}
\end{equation}
though, to avoid difficulty at $\omega = 0$, this must be re-written
\begin{equation}
	p(q,\omega) = \frac{3i}{2\nu_\textrm{max}|q|} \left[ \left(x^2 - 1\right)(2\operatorname{atanh}(x)-i\pi) - 2x \right]
\end{equation}

\end{document}
